\documentclass[12pt]{article}
\usepackage[margin=1in]{geometry}
\usepackage{amsmath,amsfonts,amssymb}
\usepackage{graphicx}
\usepackage{hyperref}
\usepackage{authblk}
\usepackage{abstract}
\usepackage{titlesec}
\usepackage{fancyhdr}
\usepackage{setspace}
\usepackage{cite}

% Header and footer
\pagestyle{fancy}
\fancyhf{}
\rhead{\thepage}
\lhead{Matthew Long | Yoneda AI}

% Title spacing
\titlespacing*{\section}{0pt}{18pt}{6pt}
\titlespacing*{\subsection}{0pt}{12pt}{4pt}

% Title format
\titleformat{\section}{\normalfont\Large\bfseries}{\thesection}{1em}{}
\titleformat{\subsection}{\normalfont\large\bfseries}{\thesubsection}{1em}{}

\title{Timelessness and Simplicity: Reformulating Physical Law Without Fundamental Time in Emergent Spacetime}
\author[1]{Matthew Long}
\author[2]{ChatGPT 4o}
\author[3]{Claude Sonnet 4}
\affil[1]{Yoneda AI}
\affil[2]{OpenAI}
\affil[3]{Anthropic}
\date{\today}

\begin{document}

\maketitle

\begin{abstract}
This paper explores how the elimination of time as a fundamental parameter leads to dramatic simplification of physical laws and provides new approaches to unification in theoretical physics. Building on emergent spacetime theories and timeless formulations of quantum mechanics, we demonstrate that treating time as emergent rather than fundamental resolves numerous conceptual difficulties in contemporary physics while pointing toward more elegant mathematical structures. We examine timeless formulations including the Wheeler-DeWitt equation, path integrals over spatial geometries, and constraint-based approaches that replace temporal dynamics with atemporal relationships. The resulting framework suggests that physical laws are most naturally expressed as constraints on possible configurations rather than rules governing temporal evolution. This shift from dynamics to constraint satisfaction enables new approaches to quantum gravity, cosmology, and the foundations of physics that may prove more fundamental and mathematically natural than traditional time-dependent formulations.
\end{abstract}

\onehalfspacing

\tableofcontents

\newpage

\section{Introduction}

\subsection{The Problem of Time in Physics}

Time occupies a peculiar position in contemporary physics. While temporal evolution appears central to physical processes as we experience them, the deepest theories of physics increasingly suggest that time itself may not be fundamental. This tension creates numerous conceptual and technical difficulties that have hindered progress in theoretical physics for decades.

Consider the contrasts:
\begin{itemize}
\item Classical mechanics treats time as an absolute parameter, while general relativity shows time to be relative and dynamical
\item Quantum mechanics appears to require time for evolution via the Schrödinger equation, yet quantum gravity theories like the Wheeler-DeWitt equation contain no time parameter
\item Thermodynamics defines temporal direction through entropy increase, yet fundamental microphysics is time-reversible
\item Cosmology describes universal temporal evolution, yet eternal inflation suggests our temporal sequence may be merely local
\end{itemize}

\subsection{The Promise of Timeless Physics}

We propose that these difficulties resolve when time is understood as emergent rather than fundamental. Timeless formulations of physics offer several advantages:

\textbf{Conceptual Clarity}: Eliminating time as a fundamental parameter removes numerous conceptual puzzles about the nature of temporal flow, the reality of past and future, and the relationship between physical and psychological time.

\textbf{Mathematical Simplicity}: Timeless formulations often exhibit greater mathematical elegance, replacing complex dynamical equations with simpler constraint relationships.

\textbf{Unification Potential}: Treating time as emergent provides new avenues for unifying quantum mechanics and general relativity by placing both theories within a common timeless framework.

\textbf{Computational Advantages}: Timeless approaches may be more amenable to computational implementation, replacing sequential simulation with constraint satisfaction.

This paper develops these ideas systematically, showing how timeless physics leads to simpler, more fundamental formulations of physical law.

\section{Historical Foundations}

\subsection{Classical Antecedents}

The idea that time might not be fundamental has deep historical roots:

\textbf{Parmenidean Eternalism}: Parmenides argued that change and temporal becoming are illusions, with reality consisting of an eternal, unchanging unity. This anticipates modern block universe theories where all times are equally real.

\textbf{Platonic Timelessness}: Plato's distinction between the eternal realm of Forms and the temporal world of appearances suggests that the most fundamental aspects of reality transcend temporal sequence.

\textbf{Augustinian Time}: Augustine's analysis of time in the \emph{Confessions} revealed conceptual difficulties with treating time as fundamental, anticipating modern problems with the reality of temporal flow.

\textbf{Kantian Idealism}: Kant's treatment of time as a form of intuition rather than a feature of things-in-themselves prefigures modern views of time as emergent from more fundamental structures.

\subsection{Early Modern Developments}

Several developments in early modern physics pointed toward timeless approaches:

\textbf{Lagrangian Mechanics}: The principle of least action replaces temporal dynamics with optimization over entire trajectories, treating temporal evolution as derived from atemporal variational principles.

\textbf{Hamiltonian Mechanics}: The Hamiltonian formulation treats time and energy symmetrically, suggesting that temporal evolution reflects constraints rather than fundamental temporal flow.

\textbf{Special Relativity}: Einstein's relativity reveals the conventional nature of simultaneity and temporal ordering, undermining absolute notions of temporal sequence.

\textbf{General Relativity}: The general theory makes time itself dynamical, suggesting that temporal structure emerges from gravitational field equations rather than providing a background framework.

\subsection{Quantum Mechanical Hints}

Quantum mechanics provided the first explicit evidence for timeless physics:

\textbf{Quantum Superposition}: Superposition states suggest that quantum systems exist in atemporal configurations that do not correspond to classical temporal sequences.

\textbf{Measurement Problem}: The measurement problem indicates that temporal evolution through unitary dynamics cannot account for measurement outcomes, suggesting that something beyond temporal evolution is required.

\textbf{Quantum Entanglement}: Entangled systems exhibit correlations that transcend spatial and temporal separation, pointing toward atemporal organizational principles.

\textbf{Path Integrals}: Feynman's path integral formulation sums over entire histories rather than following temporal evolution, treating all possible temporal sequences symmetrically.

\section{Timeless Quantum Mechanics}

\subsection{The Wheeler-DeWitt Equation}

The Wheeler-DeWitt equation represents the most explicit timeless formulation in fundamental physics:

\begin{equation}
\hat{H}|\Psi\rangle = 0
\end{equation}

This equation describes the quantum state of the universe without any time parameter. Unlike the time-dependent Schrödinger equation:
\begin{equation}
i\hbar\frac{\partial}{\partial t}|\psi\rangle = \hat{H}|\psi\rangle
\end{equation}
the Wheeler-DeWitt equation is a constraint equation that the universal wavefunction must satisfy.

The absence of time in the Wheeler-DeWitt equation is not a technical limitation but reflects the fact that time itself is part of the gravitational field that must be quantized. In quantum gravity, there is no external time parameter with respect to which the universe can evolve.

\subsection{Constraint-Based Quantum Mechanics}

The Wheeler-DeWitt equation exemplifies a constraint-based approach to quantum mechanics where physical states satisfy atemporal constraints rather than evolving through time. This approach can be generalized beyond gravity:

\textbf{Constraint Quantization}: Any classical system with constraints can be quantized using constraint operators that annihilate physical states:
\begin{equation}
\hat{C}_i|\text{phys}\rangle = 0
\end{equation}

\textbf{Reduced Phase Space}: The physical Hilbert space consists of states satisfying all constraint equations, effectively reducing the degrees of freedom to those consistent with the constraints.

\textbf{Observables}: Physical observables must commute with all constraints, ensuring that they are well-defined on the reduced phase space of physical states.

\subsection{Timeless Path Integrals}

Path integral formulations naturally accommodate timeless physics by summing over spatial geometries rather than temporal histories:

\begin{equation}
Z = \int \mathcal{D}[g,\phi] e^{iS[g,\phi]}
\end{equation}

In this expression, the integration is over spatial geometries $g$ and matter field configurations $\phi$ rather than temporal evolution histories. The action $S$ becomes a functional of spatial configurations rather than temporal sequences.

This approach treats all spatial configurations democratically, without privileging any particular temporal ordering. Time emerges as a derived concept through correlations between different spatial regions rather than as a fundamental parameter.

\section{Emergent Time from Constraint Structure}

\subsection{Time as a Relational Concept}

In timeless physics, time emerges as a relational concept rather than an absolute parameter. Following insights from relational quantum mechanics and the thermal interpretation of quantum mechanics, temporal relationships arise from correlations between subsystems rather than absolute temporal flow.

Consider two quantum systems A and B with Hamiltonians $H_A$ and $H_B$. The "time" experienced by system A relative to system B can be defined through their relative evolution:

\begin{equation}
t_{A|B} = \frac{\langle A|\hat{N}_A|A\rangle}{\langle B|\hat{N}_B|B\rangle}
\end{equation}

where $\hat{N}_A$ and $\hat{N}_B$ are number operators that count quanta of change in each system.

\subsection{Thermal Time Hypothesis}

The thermal time hypothesis, developed by Connes and Rovelli, provides a mechanism for temporal emergence from thermodynamic concepts:

\begin{equation}
\frac{d}{dt}|\psi\rangle = -i\frac{\hat{H}}{\langle H\rangle}|\psi\rangle
\end{equation}

In this formulation, time emerges as a parameter describing thermodynamic evolution rather than being imposed as a fundamental structure. The temporal parameter $t$ represents thermal equilibration rather than absolute time flow.

\subsection{Quantum Reference Frames}

Quantum reference frames provide another mechanism for temporal emergence. Instead of absolute time, we have relative temporal relationships between different quantum systems serving as clocks:

\begin{equation}
|\psi\rangle_{AB} = \sum_{t_A,t_B} c(t_A,t_B)|t_A\rangle_A \otimes |t_B\rangle_B
\end{equation}

The temporal coordinate $t_A$ represents the "time" as measured by system A, while $t_B$ represents time as measured by system B. Physical predictions depend only on relative temporal relationships, not absolute time values.

\section{Constraint Satisfaction vs. Dynamical Evolution}

\subsection{The Shift from Dynamics to Constraints}

Traditional physics formulates laws as differential equations governing temporal evolution:
\begin{align}
\frac{dx}{dt} &= f(x,t) \\
\frac{d^2x}{dt^2} &= F(x,\dot{x},t)
\end{align}

Timeless physics replaces these dynamical equations with constraint relationships:
\begin{align}
C_1(x,p) &= 0 \\
C_2(x,p) &= 0 \\
&\vdots
\end{align}

This shift has profound implications for how we understand physical law. Instead of rules governing temporal change, we have constraints that determine which configurations are physically possible.

\subsection{Hamilton-Jacobi Theory}

The Hamilton-Jacobi formulation of classical mechanics provides a bridge between dynamical and constraint-based approaches. The Hamilton-Jacobi equation:
\begin{equation}
H\left(q,\frac{\partial S}{\partial q},t\right) + \frac{\partial S}{\partial t} = 0
\end{equation}
can be rewritten as a constraint when the system is autonomous:
\begin{equation}
H\left(q,\frac{\partial S}{\partial q}\right) - E = 0
\end{equation}

This constraint determines the action function $S(q)$ without reference to temporal evolution. Classical trajectories emerge as characteristics of this constraint equation rather than as fundamental temporal processes.

\subsection{Variational Principles}

Variational principles naturally embody the constraint-based approach. The principle of least action:
\begin{equation}
\delta S = \delta\int L(q,\dot{q},t)dt = 0
\end{equation}
determines entire trajectories through extremization rather than temporal evolution rules.

In timeless formulations, variational principles become constraints on field configurations or geometric structures rather than principles governing temporal development.

\section{Simplification of Physical Laws}

\subsection{Quantum Field Theory}

Timeless formulations dramatically simplify quantum field theory by eliminating the need for time-ordering and resolving issues with vacuum energy and cosmological constants.

In conventional QFT, the vacuum energy diverges:
\begin{equation}
\langle 0|H|0\rangle = \sum_k \frac{1}{2}\hbar\omega_k = \infty
\end{equation}

Timeless formulations avoid this problem by treating the vacuum as a constraint rather than a temporal ground state. The constraint:
\begin{equation}
(\hat{H} - E_{vac})|\text{vac}\rangle = 0
\end{equation}
determines the vacuum configuration without implying infinite energy density.

\subsection{General Relativity}

Einstein's field equations become simpler when formulated as constraints on spatial geometry rather than evolution equations for spacetime:

Instead of:
\begin{equation}
G_{\mu\nu} = 8\pi T_{\mu\nu}
\end{equation}

We have constraints:
\begin{align}
\mathcal{H} &= 0 \quad \text{(Hamiltonian constraint)} \\
\mathcal{H}_i &= 0 \quad \text{(Momentum constraints)}
\end{align}

These constraints determine spatial geometries directly rather than governing their temporal evolution. The metric tensor becomes a constraint variable rather than a dynamical field.

\subsection{Quantum Gravity}

Timeless formulations provide the most natural approach to quantum gravity by avoiding the problem of time that plagues canonical quantization approaches.

The Wheeler-DeWitt equation:
\begin{equation}
\left(-\frac{\hbar^2}{2}\sqrt{g}G_{ijkl}\frac{\delta^2}{\delta g_{ij}\delta g_{kl}} + \sqrt{g}R\right)|\Psi\rangle = 0
\end{equation}
constrains the universal wavefunction without requiring a temporal framework for evolution.

This eliminates numerous technical difficulties:
\begin{itemize}
\item No need for a time parameter in quantum gravity
\item No conflict between general covariance and canonical quantization
\item No problem with defining time evolution in the absence of external clocks
\end{itemize}

\section{Unification Through Timelessness}

\subsection{Quantum Mechanics and General Relativity}

The apparent incompatibility between quantum mechanics and general relativity largely stems from their different treatments of time:

\textbf{Quantum Mechanics}: Treats time as an external parameter with respect to which quantum states evolve unitarily.

\textbf{General Relativity}: Makes time itself dynamical and observer-dependent, eliminating any absolute temporal framework.

Timeless formulations resolve this conflict by treating both quantum mechanical and gravitational phenomena as aspects of constraint satisfaction in a timeless substrate. The Wheeler-DeWitt equation provides a unified framework where both quantum and gravitational effects appear as constraints on the universal wavefunction.

\subsection{Information-Theoretic Unification}

Timeless physics naturally connects with information-theoretic approaches to fundamental physics. When time is eliminated as a fundamental parameter, information organization becomes the primary organizational principle.

The constraint structure of timeless physics can be understood information-theoretically:
\begin{itemize}
\item Constraints encode information about possible system configurations
\item Physical states represent informationally consistent configurations
\item Observables extract information that respects constraint structure
\end{itemize}

This connection suggests that timeless physics and information-theoretic approaches are complementary perspectives on the same underlying reality.

\subsection{Categorical Unification}

Category theory provides mathematical tools for understanding how different physical theories unify within timeless frameworks. Instead of temporal evolution, we have functorial relationships between categories:

\begin{equation}
F: \mathcal{C}_{QM} \to \mathcal{C}_{GR}
\end{equation}

where $\mathcal{C}_{QM}$ and $\mathcal{C}_{GR}$ are categories representing quantum mechanical and gravitational phenomena respectively.

Timeless formulations make these categorical relationships more apparent by eliminating the temporal complications that obscure the underlying structural similarities between different areas of physics.

\section{Computational Advantages}

\subsection{Constraint Satisfaction Problems}

Timeless physics reformulates physical problems as constraint satisfaction problems (CSPs) rather than temporal simulation problems. CSPs often have superior computational properties:

\textbf{Parallelization}: Constraint satisfaction can be naturally parallelized since different constraints can be checked independently.

\textbf{Backtracking}: CSP algorithms can backtrack from inconsistent partial solutions, avoiding the need to simulate entire temporal sequences.

\textbf{Global Optimization}: CSP approaches can find globally optimal solutions rather than being trapped in local minima of temporal evolution.

\subsection{Quantum Computing Applications}

Timeless formulations may be particularly well-suited for implementation on quantum computers:

\textbf{Quantum Constraint Satisfaction}: Quantum algorithms for constraint satisfaction can exploit superposition to explore multiple constraint configurations simultaneously.

\textbf{Adiabatic Quantum Computing}: Adiabatic approaches naturally implement constraint satisfaction by finding ground states of constraint Hamiltonians.

\textbf{Quantum Annealing}: Quantum annealing techniques can find optimal solutions to constraint satisfaction problems that represent physical configurations.

\subsection{Algorithmic Information Theory}

Timeless physics connects naturally with algorithmic information theory (AIT) by treating physical laws as compression algorithms for describing possible configurations:

\begin{equation}
K(x) = \min_{p}\{|p| : U(p) = x\}
\end{equation}

where $K(x)$ is the Kolmogorov complexity of configuration $x$, representing the shortest program that generates $x$.

In timeless physics, physical laws minimize the algorithmic information needed to specify consistent configurations, providing a natural connection between physical and computational principles.

\section{Cosmological Implications}

\subsection{The Beginning of Time}

Timeless cosmology eliminates conceptual difficulties with the "beginning" of time by treating temporal structure as emergent rather than fundamental. Instead of asking what happened "before" the Big Bang, we ask about the constraint structure that gives rise to emergent temporal sequences.

The Big Bang becomes a boundary condition in the space of possible configurations rather than a temporal event. The apparent initial singularity may reflect incomplete understanding of the constraint structure rather than genuine physical infinities.

\subsection{Eternal Inflation}

Eternal inflation theories naturally fit within timeless frameworks. Instead of eternal temporal sequences, we have eternal constraint structures that give rise to local temporal domains.

Different inflationary patches correspond to different solutions of the same timeless constraint equations rather than temporally separated regions of an evolving universe. This eliminates conceptual difficulties with infinite temporal sequences and past-eternal inflation.

\subsection{Multiverse and Measure Problems}

The measure problem in cosmology—how to assign probabilities to different outcomes in an eternal multiverse—becomes more tractable in timeless frameworks. Instead of temporal measures over infinite sequences, we have atemporal measures over constraint-satisfying configurations.

The typicality arguments that support anthropic reasoning can be formulated as statements about the measure of constraint-satisfying configurations rather than claims about temporal evolution in multiverse scenarios.

\section{Consciousness and Experience}

\subsection{The Phenomenology of Temporal Flow}

One of the strongest challenges to timeless physics comes from the apparent reality of temporal experience. We seem to directly experience temporal flow, change, and temporal direction in ways that appear incompatible with timeless reality.

However, phenomenological analysis reveals that temporal experience itself may be constructed rather than fundamental:

\textbf{Specious Present}: The experienced "now" always has temporal extent, suggesting that immediate experience encompasses temporal relationships rather than instantaneous temporal positions.

\textbf{Temporal Synthesis}: Conscious experience involves the synthesis of temporal information into unified experiential content, suggesting that temporal relationships are constructed rather than directly given.

\textbf{Memory and Anticipation}: Temporal experience always involves memory of "past" states and anticipation of "future" states, indicating that experience constructs temporal relationships rather than passively reflecting temporal flow.

\subsection{Consciousness as Constraint Satisfaction}

Integrated Information Theory (IIT) suggests that consciousness corresponds to integrated information—information that cannot be decomposed into independent parts. This naturally fits with constraint-based approaches to timeless physics.

Conscious experience may represent the solution to constraint satisfaction problems involving the integration of information across different scales and domains. The apparent temporal flow of consciousness becomes the result of solving increasingly complex constraint satisfaction problems rather than temporal evolution.

\subsection{Semantic Coherence}

In information-theoretic approaches to consciousness, conscious experience corresponds to semantic coherence rather than temporal sequence. Meanings and intentional contents achieve coherence through constraint satisfaction rather than temporal development.

This connects consciousness naturally with timeless physics: both involve the satisfaction of constraints that determine coherent structures rather than temporal evolution processes.

\section{Philosophical Implications}

\subsection{The Nature of Physical Law}

Timeless physics fundamentally alters our understanding of what physical laws are. Instead of rules governing temporal change, physical laws become specifications of possibility—constraints that determine which configurations can exist.

This shift has several important implications:

\textbf{Modal Rather Than Temporal}: Physical laws specify what is possible rather than what happens temporally.

\textbf{Structural Rather Than Causal}: Laws describe structural relationships rather than causal processes.

\textbf{Logical Rather Than Dynamical}: Physical constraints resemble logical constraints more than mechanical processes.

\subsection{Causation and Explanation}

Traditional notions of causation rely heavily on temporal sequence: causes precede effects, and causal explanation involves showing how earlier states bring about later states. Timeless physics requires reconceptualizing causation and explanation.

\textbf{Structural Causation}: Causal relationships become structural relationships within constraint-satisfying configurations rather than temporal sequences.

\textbf{Constraint-Based Explanation}: Explanation involves showing how phenomena satisfy relevant constraints rather than deriving them from temporal antecedents.

\textbf{Teleological Elements}: Constraint satisfaction can exhibit teleological aspects where configurations are determined by global optimization rather than local temporal processes.

\subsection{Free Will and Determinism}

The traditional debate between free will and determinism assumes temporal frameworks where present choices affect future outcomes. Timeless physics reframes this debate entirely.

In timeless frameworks, the apparent tension between free will and determinism dissolves:
\begin{itemize}
\item There are no temporal sequences for deterministic laws to govern
\item Choice becomes a matter of which constraint-satisfying configurations are actualized
\item Freedom and determinism become complementary aspects of constraint satisfaction
\end{itemize}

\section{Future Research Directions}

\subsection{Mathematical Development}

Several areas of mathematics need further development to support timeless physics:

\textbf{Constraint Theory}: Developing systematic mathematical frameworks for understanding constraint satisfaction in physical systems.

\textbf{Categorical Dynamics}: Understanding how apparent temporal evolution emerges from categorical relationships between atemporal structures.

\textbf{Information Geometry}: Connecting constraint satisfaction with information-geometric approaches to understanding physical structure.

\subsection{Empirical Consequences}

Timeless physics needs to make testable predictions to gain empirical support:

\textbf{Quantum Gravity Phenomenology}: Testing whether quantum gravity effects exhibit signatures of timeless rather than temporal structure.

\textbf{Cosmological Predictions}: Developing cosmological tests that distinguish between temporal and timeless approaches to early universe physics.

\textbf{Laboratory Tests}: Investigating whether quantum mechanical systems exhibit evidence of timeless constraint structure in precision experiments.

\subsection{Computational Implementation}

Developing computational tools for timeless physics represents a crucial research direction:

\textbf{Constraint Solvers}: Adapting advanced constraint satisfaction algorithms to physical problems.

\textbf{Quantum Algorithms}: Developing quantum algorithms specifically designed for constraint satisfaction in physical systems.

\textbf{Simulation Environments}: Creating computational environments that implement timeless physics principles directly rather than through temporal approximation.

\section{Conclusion}

We have argued that eliminating time as a fundamental parameter leads to dramatic simplification and unification in theoretical physics. Key conclusions include:

\textbf{Conceptual Clarity}: Timeless formulations resolve numerous conceptual puzzles that plague time-dependent approaches to fundamental physics.

\textbf{Mathematical Elegance}: Constraint-based formulations often exhibit greater mathematical simplicity and naturalness than their temporal counterparts.

\textbf{Unification Potential}: Timeless approaches provide new avenues for unifying quantum mechanics, general relativity, and information theory within common mathematical frameworks.

\textbf{Computational Advantages}: Reformulating physics as constraint satisfaction may provide superior computational approaches for both classical and quantum systems.

\textbf{Empirical Consistency}: Timeless physics appears consistent with existing empirical evidence while potentially making new testable predictions.

The transition from temporal dynamics to atemporal constraint satisfaction represents a fundamental shift in how we conceptualize physical reality. This shift may prove as significant as the transitions from Aristotelian to Newtonian physics or from classical to quantum mechanics.

The simplicity and elegance of timeless formulations suggest that time, like absolute space before it, may be a useful approximation rather than a fundamental feature of reality. As physics continues to probe deeper levels of reality, timeless approaches may provide the conceptual and mathematical tools needed for the next major advances in our understanding of the natural world.

The ultimate goal is not merely theoretical elegance but practical understanding: developing physical theories that are both mathematically natural and empirically successful. Timeless physics appears to offer promising directions for achieving both objectives simultaneously.

\section*{Acknowledgments}

The author thanks researchers working on timeless approaches to quantum gravity, foundations of quantum mechanics, and constraint-based physics whose work provides the foundation for these investigations. Special appreciation goes to those developing the Wheeler-DeWitt equation, constraint quantization methods, and emergent spacetime approaches whose technical contributions make this synthesis possible.

\section*{References}

\begin{thebibliography}{99}

\bibitem{dewitt1967}
DeWitt, B.S. (1967). Quantum theory of gravity. I. The canonical theory. \emph{Physical Review}, 160(5), 1113-1148.

\bibitem{barbour1999}
Barbour, J. (1999). \emph{The end of time: The next revolution in physics}. Oxford University Press.

\bibitem{rovelli2004}
Rovelli, C. (2004). \emph{Quantum gravity}. Cambridge University Press.

\bibitem{connes1994}
Connes, A., \& Rovelli, C. (1994). Von Neumann algebra automorphisms and time-thermodynamics relation in generally covariant quantum theories. \emph{Classical and Quantum Gravity}, 11(12), 2899-2917.

\bibitem{page1983}
Page, D.N., \& Wootters, W.K. (1983). Evolution without evolution: Dynamics described by stationary observables. \emph{Physical Review D}, 27(12), 2885-2892.

\bibitem{wheeler1967}
Wheeler, J.A., \& DeWitt, B.S. (1967). \emph{Battelle Rencontres: 1967 Lectures in Mathematics and Physics}. W.A. Benjamin.

\bibitem{ashtekar2004}
Ashtekar, A. (2004). Background independent quantum gravity: A status report. \emph{Classical and Quantum Gravity}, 21(15), R53-R152.

\bibitem{smolin2013}
Smolin, L. (2013). \emph{Time reborn: From the crisis in physics to the future of the universe}. Houghton Mifflin Harcourt.

\bibitem{penrose2004}
Penrose, R. (2004). \emph{The road to reality: A complete guide to the laws of the universe}. Oxford University Press.

\bibitem{hawking1973}
Hawking, S.W. (1973). The large scale structure of space-time. Cambridge University Press.

\bibitem{feynman1948}
Feynman, R.P. (1948). Space-time approach to non-relativistic quantum mechanics. \emph{Reviews of Modern Physics}, 20(2), 367-387.

\bibitem{dirac1958}
Dirac, P.A.M. (1958). The theory of gravitation in Hamiltonian form. \emph{Proceedings of the Royal Society of London A}, 246(1246), 333-343.

\bibitem{arnowitt1962}
Arnowitt, R., Deser, S., \& Misner, C.W. (1962). The dynamics of general relativity. In \emph{Gravitation: An Introduction to Current Research}, 227-265.

\bibitem{kuchař1992}
Kuchař, K.V. (1992). Time and interpretations of quantum gravity. In \emph{Proceedings of the 4th Canadian Conference on General Relativity and Relativistic Astrophysics}, 211-314.

\bibitem{isham1993}
Isham, C.J. (1993). Canonical quantum gravity and the problem of time. In \emph{Integrable Systems, Quantum Groups, and Quantum Field Theories}, 157-287.

\bibitem{anderson2012}
Anderson, E. (2012). \emph{The problem of time: Quantum mechanics versus general relativity}. Springer.

\bibitem{gell-mann1993}
Gell-Mann, M., \& Hartle, J.B. (1993). Classical equations for quantum systems. \emph{Physical Review D}, 47(8), 3345-3368.

\bibitem{halliwell1991}
Halliwell, J.J. (1991). Quantum mechanical histories and the uncertainty principle. \emph{Physical Review D}, 44(8), 2358-2374.

\bibitem{griffiths1984}
Griffiths, R.B. (1984). Consistent histories and the interpretation of quantum mechanics. \emph{Journal of Statistical Physics}, 36(1-2), 219-272.

\bibitem{omnès1992}
Omnès, R. (1992). Consistent interpretations of quantum mechanics. \emph{Reviews of Modern Physics}, 64(2), 339-382.

\bibitem{hartle1993}
Hartle, J.B. (1993). The quantum mechanics of cosmology. In \emph{Quantum Cosmology and Baby Universes}, 65-157.

\bibitem{vilenkin1982}
Vilenkin, A. (1982). Creation of universes from nothing. \emph{Physics Letters B}, 117(1-2), 25-28.

\bibitem{linde1983}
Linde, A.D. (1983). Chaotic inflation. \emph{Physics Letters B}, 129(3-4), 177-181.

\bibitem{guth1981}
Guth, A.H. (1981). Inflationary universe: A possible solution to the horizon and flatness problems. \emph{Physical Review D}, 23(2), 347-356.

\bibitem{tegmark2003}
Tegmark, M. (2003). Parallel universes. \emph{Scientific American}, 288(5), 40-51.

\bibitem{weinberg1989}
Weinberg, S. (1989). The cosmological constant problem. \emph{Reviews of Modern Physics}, 61(1), 1-23.

\bibitem{bousso2000}
Bousso, R. (2000). The holographic principle for general backgrounds. \emph{Classical and Quantum Gravity}, 17(5), 997-1005.

\bibitem{susskind1995}
Susskind, L. (1995). The world as a hologram. \emph{Journal of Mathematical Physics}, 36(11), 6377-6396.

\bibitem{maldacena1998}
Maldacena, J. (1998). The large N limit of superconformal field theories and supergravity. \emph{Advances in Theoretical and Mathematical Physics}, 2(2), 231-252.

\end{thebibliography}

\end{document}