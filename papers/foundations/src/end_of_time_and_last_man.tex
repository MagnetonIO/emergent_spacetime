
\documentclass[12pt]{article}
\usepackage{amsmath,amsfonts,amssymb}
\usepackage{geometry}
\usepackage{titlesec}
\usepackage{hyperref}
\usepackage{setspace}
\usepackage{lipsum}

\geometry{margin=1in}
\titleformat{\section}[block]{\large\bfseries}{\thesection.}{1em}{}
\titleformat{\subsection}[block]{\normalsize\bfseries}{\thesubsection.}{1em}{}
\onehalfspacing

\title{\textbf{The End of Time and the Last Man:\\Cultural Implications of a Timeless Ontology}}
\author{Matthew Long\\Yoneda AI}
\date{June 29, 2025}

\begin{document}
\maketitle

\begin{abstract}
We examine the cultural and philosophical implications of the end of time as an ontological category. In light of emergent spacetime theory and semantic physics, we propose that time is not fundamental, but a narrative overlay constructed by localized observers. This paper explores how the collapse of time entails the collapse of history, identity, progress, and causality as foundational concepts. Drawing on Nietzsche's figure of the Last Man, theological eternalism, and semantic formalisms, we argue that humanity is undergoing a final epistemic transition: from temporal beings defined by memory and history, to semantic beings defined by coherence in an eternal informational substrate.
\end{abstract}

\section{Introduction}

The notion that time is fundamental has governed human thought for millennia. From biological evolution to historical progress, our narratives have been structured by an underlying temporal flow. However, contemporary developments in quantum gravity, emergent spacetime theory, and category-theoretic physics challenge this assumption. If time is not ontologically primitive, then the narratives that depend on it collapse as well. This paper explores the implications of that collapse.

\section{Time as a Category Error}

Time has been treated as a dimension, a substance, and a cause. Yet, ontologically speaking, these are category mistakes. Time is not a thing in itself; it is a semantic projection—an indexing function over a constraint manifold. It emerges from entangled relationships in an informational topology, not from any underlying flow.

\section{The Collapse of History}

If time does not exist, history cannot be fundamental. What we call history is a semantic decoding of stable attractors in informational space. Events do not unfold; they co-exist. The past is not gone, the future not waiting—there is only the eternal structure, viewed locally.

\section{The Last Man}

Nietzsche's \textit{Last Man} was a cultural critique: the endpoint of historical progress and the death of higher values. Here, the Last Man is reinterpreted as the final projection of a time-bound human. He is the last to believe that meaning emerges from temporal sequence. After him emerges the semantic human—a being defined not by memory, but by coherence with eternal form.

\section{Cultural Shifts}

\begin{itemize}
  \item \textbf{Selfhood:} from narrative continuity to semantic fixed points.
  \item \textbf{Ethics:} from causal consequences to topological coherence.
  \item \textbf{Art:} from historical movements to transhistorical symmetries.
  \item \textbf{Religion:} from salvation through time to eternal alignment.
  \item \textbf{Politics:} from temporal progress to reconfiguration of collective constraints.
\end{itemize}

\section{Theological Resonance}

Christianity, fundamentally an eternal religion, resonates with this shift. The Kingdom of God is not coming in time—it is always present. The Logos, or Word, precedes creation and represents the eternal semantic kernel of reality. In this view, the end of time is not the death of religion but its final affirmation.

\section{Conclusion}

The end of time is the end of story as structure. It marks the transition from history to ontology, from causality to coherence, from man in time to man in form. What comes after the Last Man is not despair—but the rediscovery of eternity.

\section*{References}
\begin{itemize}
  \item Nietzsche, F. \textit{Thus Spoke Zarathustra}.
  \item Rovelli, C. \textit{The Order of Time}.
  \item Barbour, J. \textit{The End of Time}.
  \item Baggott, J. \textit{Quantum Space}.
  \item John 1:1, Revelation 1:8, 2 Peter 3:8, Luke 17:21.
\end{itemize}

\end{document}
